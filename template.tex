\documentclass{article}
\usepackage{arxiv}
\usepackage[utf8]{inputenc} % allow utf-8 input
\usepackage[T1]{fontenc}    % use 8-bit T1 fonts
\usepackage{hyperref}       % hyperlinks
\usepackage{url}            % simple URL typesetting
\usepackage{booktabs}       % professional-quality tables
\usepackage{amsfonts}       % blackboard math symbols
\usepackage{nicefrac}       % compact symbols for 1/2, etc.
\usepackage{microtype}      % microtypography
\usepackage{lipsum}
\usepackage{hyperref}

\usepackage{amsmath, amsfonts, amsthm,amssymb}
\usepackage[english, russian]{babel}
\usepackage{graphicx}

\newtheorem{theorem}{Теорема}
\newtheorem{lemma}{Лемма}
\newtheorem{corollary}{Следствие}
\newtheorem{proposition}{Предложение}
\newtheorem{assertion}{Утверждение}

% \title{Theoretical Analysis of Random Geometric Hierarchical K-Graph}
% \title{NP-completeness and LP formulation of Dota Underlords problem}
 \title{Hacking Dota Underlords With Integer Programming Model}


\author{
  Alexander A. ~Ponomarenko, Dmitry S. ~Sirontkin\thanks{Use footnote for providing further
    information about author (webpage, alternative
    address)---\emph{not} for acknowledging funding agencies.} \\
  Laboratory of Algorithms and Technologies for Network Analysis\\
  National Research University Higher School of Economics \\
  Nizhny Novgorod, Russia\\
  \texttt{aponomarenko@hse.ru, dsirotkin@hse.ru} \\
  %% examples of more authors
  %% \AND
  %% Coauthor \\
  %% Affiliation \\
  %% Address \\
  %% \texttt{email} \\
  %% \And
  %% Coauthor \\
  %% Affiliation \\
  %% Address \\
  %% \texttt{email} \\
  %% \And
  %% Coauthor \\
  %% Affiliation \\
  %% Address \\
  %% \texttt{email} \\
}
%

\begin{document}
\maketitle

\begin{abstract}
В данной работе мы демонстрируем, как задача оптимального выбора состава команды в популярной компьютерной игре Dota UnderLords может быть сведелна к задаче целочисленного линейного программирования. Мы приводим модель и её решения. Также мы доказываем, что данная задача относится к классу NP-hard. 
\end{abstract}


% keywords can be removed
\keywords{Целочисленное программирование \and NP-трудность \and NP-полнота}


\section{Введение}

Люди любят играть в игры. Многие игры и головоломки, в которые люди играют, интересны из-за их сложности: для их решения требуется сообразительность. Часто эта трудность может быть показана математически в виде функции вычислительной сложности от размера входного набора данных. Например, было показано, что шахматы относятся к классу EXPTIME \cite{fraenkel1981computing}; задача выбора игрока  в легендарной видео-игре "Тетрис" относится классу NP-трудных \cite{breukelaar2004tetris}. Было показано, что игра-головомка "Сокобан" ("кладовщик", также известная для русскоговоряшего населения под названием "мудрый крот" ) полиномиально разрешима \cite{hearn2005pspace}.

Особое место в теории сложности вычислений занимает класс NP-полных задач (класс NP-complete). %Ранее было показано, что к классу NP-comple относится много известных задач. 
%Также было недавно было показано, что некоторые популярные игры тоже относятся к этому классу.
Было показано  \cite{kaye2000minesweeper}, что к классу NP-полных задач относится игра "Сапёр". Задача нахождения решения требующего минимального количество передвижений фишек в обощеной версии игры 15-ки для доски размером $N \times N$ тоже относится к классу NP-complete \cite{ratner1986finding}. 
В некотором смысле каждая NP-полная задача является загадкой, и наоборот, многие головоломки NP-полны. Для углубленого изучения темы вычислительной сложности для головоломок и игр, мы отсылаем читателя к обзору  \cite{costa2018computational}.

В данной работе мы уделяем внимание популярной видео-игре Dota UnlderLords. 
Как оказывается, данную задачу можно представить как задачу комбинаторной оптимизации, которая при фиксированном числе альяноств принадлежит классу NP-complete.

Статья организованна следующим образом. В разделе \ref{SectionDUDescription} повествуется в чём заключается игра Dota UnderLords.
Мы приводим формулировку Dota UnderLords в виде задачи линейного целочисленного программирования разделе \ref{SectionDUIP}. В разделе \ref{SectionNPCompleteProof} мы показываем  её NP-полноту. Результаты решения модели целочисленного программирования для реальных данных мы публикуем в разделе \ref{SectionComputationalResults}. И по традиции мы подводим итоги в секции  \ref{SectionConclusion} "Заключение".
 
%В данном случае, для нас самих было неожиданностью, что задача UnderLords относится классу NP-полных задач. 

\section{Описание игры Dota Underlords}
\label{SectionDUDescription}
%Here will be our great introduction that will inspire everybody who reads it.

По ходу игры восемь игроков составляют команду из <<героев>> – существ, способных сражаться друг с другом на игровой карте. У каждого из героев есть базовые параметры - здоровье, урон, скорость атаки и прочие, а также особоая способность, которая определяет его роль в игре. Каждый герой принадлежит к двум или более <<альянсам>> - классам, в которые входят несколько героев. так, например, герой Enchantress принадлежит одновременно к альянсу <<друиды>> и к альянсу <<хищники>>. При наборе нескольких героев из одного альянса (для каждого альянса это число индивидуально) игрок получает бонус, который выражается в усилении всех героев из альянса, усилении всех своих героев или ослаблении всех героев соперника. Последнее может быть интерпретировано как относительное усиление своих героев и поэтому на протяжении работы будут рассматриваться только первые два случая. Следует отметить, что для одного альянса может быть несколько бонусов, которые открываются разным количеством героев соответствующего альянса, при этом они могут быть разного типа.

Также в ходе игры можно усилять своих героев до более высоких уровней или путём покупки внутриигровых предметов. В рамках данной работы эти аспекты учитываться не будут.

Таким образом, сила команды игрока определяется как:

\begin{enumerate}
    \item Силой выбранных героев
    \item Бонусами от альянсов в которых они состоят
\end{enumerate}



\section{Перевод задачи в язык линейного целочисленного программирования }
\label{SectionDUIP}

\subsection{Простейшая постановка задачи}

Формализуем задачу следующим образом:

Будем считать, что всего у нас в есть $n$ героев на выбор. Будем считать, что сила некоторого $i$-го героя опредяляется за некоторую неотрицательную величину $s_i$. За $x_i$ будем обозначать принадлежность героя выбранной команде. Условимся, что когда $x_i = 1$, если $i$-й герой принадлежит набранной команде и $x_i=0$ в противном случае. Тогда условие того, что в команде не более, чем $m$ героев можно записать в виде $\sum_{i=1}^n x_i \leq m$. Тогда в простейшей форме данную задачу можно сформулировать следующим образом:

\begin{equation}
\begin{gathered}
    max \sum_{i=1}^n x_i s_i \\
    \sum_{i=1}^n x_i \leq m \\
    x_i \in \{0, 1\} \text{ – управляющая переменная} \\
   n, m, s_i \text{ – костанты}  \\
\end{gathered}
\end{equation}

В данной постановке задача решается элементарно – достаточно взять $n$ элементов с наибольшими весами

\subsection{Постановка задачи с альянсами}
Как было упомянуто, в <<Dota Underlords>>  каждый герой принадлежит к двум или более <<альянсам>> - классам, в которые входят несколько героев.  Когда в команде присутствует несколько героев из одного альянса (для каждого альянса это число индивидуально) игрок получает бонус, который выражается в усилении всех героев из альянса, усилении всех своих героев или ослаблении всех героев соперника. Последнее может быть интерпретировано как относительное усиление своих героев и поэтому на протяжении работы будут рассматриваться только первые два случая. Следует отметить, что для одного альянса может быть несколько бонусов, которые открываются разным количеством героев соответствующего альянса, при этом они могут быть разного типа.

%В рамках нашей модели мы рассмотрим два типа альянсов – те которые дают бонусы своим членам и те, которые дают бонусы всем героям игрока. Пусть всего есть $t$ альянсов.

%В <<Dota Underlords>> для активации бонуса иногда можно собрать не весь альянс, а некоторую его часть – треть или половину. Более, того, могут существовать несколько подобных бонусов, например, один бонус за собранную треть альянса и бонус за собранные две трети альянса. Другим подобным случаем является случай, когда в игре присутствует больше героев в некотором альянсе, чем требуется для активации бонуса. 
%Мы будем считать, что бонус от альянса увеличивает силу героя на некоторую величину $e_{ijk}$, где $i$ -- номер героя, а $j$ -- номер альянса, $k$ -- номер уровня бонуса. 
Данную ситуцию мы предлагаем моделировать с помощью введения 3-х индексного тензора $e_{ijk} \in \mathbb{R}$ означающего бонус герою $i$, за альянс $j$, в котором присутствуют не меннее $k$ героев альянса $j$. Другими словами, $e_{ijk}$ это $k$-й бонус альянса $j$ для героя $i$.

С помошью тензора $e_{ijk}$ можно поддерижвать два типа альянсов – те которые дают бонусы своим членам и те, которые дают бонусы всем героям игрока. При этом, альянсы рассмотренных видов отличются только тем, что в альянсах, дающих бонус своим членам, величина $e_{ijk}$ равна нулю тогда и только тогда, когда $i$-й герой не принадлежит $j$-му альянсу. Заметим, что тензор  $e_{ijk}$  будет сильно разрежан, поскольку альянсы от которых бонусы идут всем гером не много.  
Контролировать вхождения бонуса $e_{ijk}$ в общую силу команды предлагается с помощью управляющей бинарной переменной $I_{ijk}$.
Так мы можем записать целевую функцию как следующую сумму $\sum_{i=1}^{n} x_i s_i + \sum_{i=1}^{n} \sum_{j=1}^{t} $.
Связь переменных $x_{i}$ и $I_{ijk}$ задаётся неравенствами
$\forall{i,j,k} :  \sum_{i'=1}^{n} a_{i'j} x_{i'} - k \ge M( I_{ijk}  - 1)$. \textbf{Здесь надо вставить пару предложений про эти неравенства, в чём логика этой связи}
Они недают бинарной переменной $I_{ijk}$ принимать значение 1, если в решение входит меньше чем $k$ героев входящих в альянс $j$. Когда решение содержит героев из альянса $j$ меньше чем $k$, левая часть этого неравенства отрицательная, поэтому чтобы неравентсва соблюдались правая часть должна быть ещё меньше. Такое возможно только, когда бинарная $I_{ijk}$ равно нулю. В этом случае правая часть равна $-М$, где $М$ константа заведомо большая, чем $k$, то есть больше, чем максимальный размер альянса $q$.
Разумно требовать, чтобы бонус для героя $i$ мог быть активирован ($I_{ijk} = 1$ ), только когда герой $i$ входит в решение. Это задаётся неравенствами $\forall{i,j,k} :  I_{ijk}  \le x_i$. Также мы хотим, чтобы бонус $e_{ijk}$ был активирован, только для если герой $i$ принадлежит альянсу $j$. Для этого мы в модель включили неравенства $\forall{i,j,k} :  I_{ijk}  \le a_{ij}$.   


Таким образом, после введения в модель альянсов, она выглядит следующим образом.
%Общая система уравнений выглядит следующим образом:

\begin{equation}
\label{DUIP}
\begin{gathered}
\textbf{Целевая функция:}\\
max \sum_{i=1}^{n} x_i s_i + \sum_{i=1}^{n} \sum_{j=1}^{t}  \sum_{k=1}^{q} e_{ijk} I_{ijk} \\
\textbf{Ограничения на входные данные}\\
\forall{j} : \sum_{i=1}^n a_{ij} \le q \\
\textbf{Ограничения на управляющие переменные} \\
\forall{i,j,k} :  \sum_{i'=1}^{n} a_{i'j} x_{i'} - k \ge M( I_{ijk}  - 1) \\
\sum_{i=1}^n x_i \le m   \\ 
% \sum_{i=0}^{n-1} a_{ij} x_{i} < k \\ 
\forall{i,j,k} :  I_{ijk}  \le x_i \\
%\forall{i,j,k} :  I_{ijk}  \le a_{ij} \\
\text{Управляющие переменные:} \\
I_{ijk} \in \{0, 1\} \text {, 1 – если для героя } i \text{, активирован} k\text{-й бонус } j \text{-го альянса,} \\
x_i  \in \{0, 1\} \text{, 1 -- если герой } i \text{– входит в решение} \\
\textbf{Константы:} \\
n \in \mathbb{N} \text{ -- число героев,} \\
m \in \mathbb{N} \text{ -- максимальный размер команды}\\
t \in \mathbb{N} \text{ -- общее количество альянсов} \\
q \in \mathbb{N} \text{ -- максимальный размер одного альянса,} \\
s_i  \in \mathbb{R} \text{–- сила героя } i, \\
e_{ijk} \in \mathbb{R} \text{ -- бонус для героя } i \text{,  если активирован } k
\text{-й бонус } j \text{-го альянса} \\
a_{ij} \in \{0, 1\} \text{, 1 -- если герой } i \text{ входит в альянс } j \\ 
\end{gathered}
\end{equation}

\section{Доказательство NP-трудной задачи Dota Underlords с альянсами}
\label{SectionNPCompleteProof}

Чтобы доказать, что задача $T$ является NP-трудной достаточно показать, что  к ней может быть сведена хотя бы одна из задач про которые известно, что она является NP-трудной.  
%Мы нашли два способа как это сделать. Первый способ – свести классичекую задачу о рюкзаке к нашей задаче. Второй –к задаче UnderLoadrs  свести задачу поиска максимального плотного подграфа. 


%\subsection{Первый способ – Сведение задачи о рюкзаке к UnderLoards}
%Классическая задача о рюкзаке формулируется следующим образом. Задан набор предметов $\hat{i}=1...\hat{n}$ для которых известны веса $w_{\hat{i}}$ и стоимости $c_i$. Требуется найти набор предметов с максимальной суммарной стоимостью и при этом, чтобы общий вес предметов не превышал максимальный заданныей вес рюкзака $W$.
%Мы строим входные данные для UnderLoards следующим образом. 
%\begin{enumerate}
%\item Альянсов столько сколько предметов. $t:=\bar{n}$
%\item Каждому предмету $\hat{i}$ мы ставим в соответствие альянс $\hat{i}$
%\item Альянс $\hat{i}$ состоит из $w_{\hat{i}}$ героев. $\forall i : \sum_j^t{a_{ij} } = w_{\hat{i}}  $
%\item Сила каждого героя равна нулю ( $\forall i : s_i = 0 $ ). 
%\item Каждый герой входит только в один альянс ($\forall i : \sum_{j=1}^t {a_{ij}} =1 $ ).
%\item Матрица $e_{ijk}$ устроена таким образом, что альянс $\hat{i}$  вносит вклад в общую силу команды только при полностью собранном альянсе, и этот вкад от альянса равен стоимости предмета  $c_{\hat{i}}$. Говоря формальным языком, матрица $e_{ijk}$ должна удовлетворять ограничению $ \forall \hat{i}, k = \textit{размер альянса }\hat{i} : \sum_{i=1}^n {e_{i \hat{i}k } } $ . 
%\end{enumerate}
%Таким образом, в оптимальном решении будут присутствовать только полностью собраные альянсы. Альянсы вошедшие в решение, будут соответстовать оптимальному набору предметов в задаче о ранце.

\subsection{Сведение задачи о поиска максимальго плотного подграфа к UnderLoards}

Расcмотрим её частный случай - пусть все альянсы имеют размер равный двум, и сила всех героев одинакова. Рассмотрим частный случай задачи Dota Underlords со следующими ограничениями:

\begin{enumerate}
    \item Силы всех героев одинаковы ($\forall i, j s_i=s_j$)
    \item Альянсы могут давать бонусы исключительно героям в них состоящим ($\forall i, j, k a_{ij}=0 \Longrightarrow e_{ijk} = 0$)
    \item Все альянсы имеют одинаковый размер, равный двум ($\forall j \sum_i a_{ij}=2$)
    \item Все альянсы дают бонус исключительно при наличии в них обоих героев ($\forall i, j e_{ij1}=0$)
    \item Бонусы ото всех альянсов равны ($\forall i, j, i', j' a_{ij}=1, a_{i' j'}=1 \Longrightarrow e_{ij2}=e_{i' j' 2}$)
\end{enumerate}

Тогда данные можно представить в виде графа $G(V, E)$, где множество вершин $V$ соответствует героям, а множество рёбер $E$ - активным альянсам. Задача поиска оптимальной команды размера $k$ таким образом сводится к поиску порождённого графа $G$ на $k$ вершинах с максимальной плотностью. Под плотностью в данной формулировке может понимается величина $\frac{G'(E)}{G'(V)}$. Действительно, при данных ограничениях общая сила команды линейно зависит от количества активных альянсов, что соответствует $G'(E)$. Т.к. $k$ неизменно, то с ростом плотности графа $G'$ растёт итоговая сила команды.

В работе \cite{downey1995fixed} было показано, что задача поиска порождённого подграфа с максимальной плотностью и фиксированным размером в графе является NP-полной. Как было показано, она является частным случаем задачи Dota Underlords и к ней сводится. Отсюда следует её NP-сложность.

\subsection{Доказательство NP-полноты - сведение UnderLords к NP-complete}
Покажем, что задача Dota Underlords сводится к задаче о максимальной взвешенной по рёбрам клике.
Из NP-плносты decision version задачи о взвешенной клике следует NP-полнота decision version задачи Dota Underlords. В данном сведении мы будем рассматривать задачу, в которой, максимальный размер альянса q ограничен константой некоторой константой.

Доказательство будет проводится последовательно, через серию теорем, каждая из которых описывает сведение всё менее и менее ограниченной версии задачи к задаче о максимальной взвешенной по рёбрам клике.

%То есть наша цель в том, чтобы постоить граф таким образом, чтобы в нем максимальная взвешенная рёберная клика (MEWC) соотвествовала оптимальному решению задачи Dota Underloars (DU).

%В графе мы будем выделять $q$ множеств вершин $\mathcal{F}_1, \mathcal{F}_2, ..., \mathcal{F}_q$, где $q$ – максимальный возможноый размер альянса.   
%
%Мы будем ассоциировать с каждой вершиной $v_i$ из множества $ \mathcal{F}_1 = \{v_1, v_2, ...v_n \}$  мы будем ставить в соответствие героя $i$. 
%
%Сила героев распределенна по множеству рёбер. Сила героя $i$ заключается в рёбрах смежных с вершниной $v_i$, соответствующей этому герою.
%Факт образования  альянса моделируется введением дополнительных вершин соответсвующих всевозможным сочетаниям героев в альянсе.
%
%Бонус за одновременное присутсвие в команде героев ($ множесто S $ ) из одного альнса  для героя   $c$, также входящего в состав команды, располагается на ребре $(v_c, V^S )$. При этом герой $c$ может как принадлежать множеству $S$, так и не принадлежать (случай, когда бонус от альянса распространяется на героев вне альянса. например, на всю команду). 

\begin{theorem}
    Задача Dota Underlords без альянсов сводится к задаче о максимальной взвешенной по рёбрам клике.
    \label{trivial_case}
\end{theorem}

\begin{proof}
    Построим граф $G$ со взвешенными рёбрами такой, что из решения задачи MEWC следует решение задачи DU. При этом размер задачи MEWC ограничен полиномом от размера задачи DU. 
    
    Построим множество $V^1$ из $n$ вершин, соответствующее множеству героев в задаче DU. Каждой вершине поставим соответствие одного из героев из задачи DU --- или, иначе говоря, назовём каждую вершину в честь одного из героев задачи DU. Пронумернуем эти вершины соответственно порядку героев $v_1^1$, $v_2^1$, ..., $v_n^1$
    
    Построим дополнительно $m-1$ множеств вершин $V^2$, $V^3$ и так далее до $V^m$, в каждом из которых также назовём по одной вершине в часть одного из героев задачи DU. Аналогично первому множеству, пронумеруем вершины в множестве $V^i$ как $v_1^i$, $v_2^i$, ..., $v_n^i$.
    
    Обозначим семейство этих множеств за $\mathcal{F}$. Таким образом мы получим $m$ множеств по $n$ вершин каждое, приэтом в каждом множестве есть по одной вершине, соотвествующей каждому из героев.
    
    Проведём рёбра следующим образом --- в графе между вершинами $v_a^i$ и $v_{a'}^{i'}$ проводится ребро, если выполняются оба следующих условия:
    \begin{itemize}
        \item Вершины $v_a^i$ и $v_{a'}^{i'}$ соответствуют разным героям ($a \neq a'$)
        \item Вершины $v_a^i$ и $v_{a'}^{i'}$ лежат в разных множествах из семейства $F$ ($i \neq i'$)
    \end{itemize}
    
    Расммотрим все максимальные клики в данном графе. Очевидно, что в любой такой клике есть ровно по одной вершине из каждого из множеств $V^i$ --- всего $m$ вершин. Также все эти вершины соответствуют разным героям. Таким образом каждая клике задаёт некоторую команду из героев в задаче DU. При этом стоит отметить, что каждой команде может соответствовать несколько клик.
    
    Теперь расставим на рёбрах веса. Для этого ребру, соединяющем вершины $s_a^i$ и $s_{a'}^{i'}$ припишем вес $\frac{s_a}{m-1} + \frac{s_{a'}}{m-1}$. Покажем, что сумма весов рёбер в клике, соответствующей некоторой команде есть в точности сила этой команды. 
    
    Действительно, в клике для каждой её вершины есть ровно $m-1$ ребро, ей инцедентное. Тогда каждое слагаемое $\frac{s_i}{m-1}$ соответствующее вершине с нижним индексом $i$ входит в сумму ровно $m-1$ раз. Отсюда следует, что сумма всех весов рёбер в клике есть сумма всех величин $s_i$, соответствующих номерам вершин, образующим данную клику.
\end{proof}

\begin{theorem}
    Задача Dota Underlords c альянсами размера ровно 2, сводится к задаче о максимальной взвешенной по рёбрам клике.
\end{theorem}

\begin{proof}
    Построим граф $G'$ со взвешенными рёбрами такой, что из решения задачи MEWC следует решение задачи DU. При этом размер задачи MEWC ограничен полиномом от размера задачи DU.
    
    Возьмём в качестве основы для построения граф $G$ из теоремы \ref{trivial_case}.
    
    Построим множество $W^{12}$ из $\binom{n}{2}$ вершин, где каждая вершина соответствует неупорядоченной паре героев. Пронумеруем эти вершины в лексикографическом порядке, соответственно порядку пар $w_{12}^1$, $w_{13}^1$, ..., $w_{n-1 n}^1$
    
    Построим дополнительно $\binom{m}{2}-1$ множеств вершин $W^{13}$, $W^{14}$ и так далее до $W^{m-1 m}$, в каждом из которых также сопоставим по вершине неупорядоченной паре героев задачи DU. Аналогично первому множеству, пронумеруем вершины в множестве $V^{i j}$ как $w_1^{i j}$, $w_2^{i j}$, ..., $w_{n-1 n}^{i j}$.
    
    Обозначим семейство этих множеств за $\mathcal{F}_2$. Таким образом мы получим $\binom{m}{2}$ множеств по $\binom{n}{2}$ вершин каждое, при этом в каждом множестве есть по одной вершине, соотвествующей каждой паре героев.
    
    Проведём дополнительные рёбра следующим образом --- в графе между вершинами $w_{ab}^{i j}$ и $w_{ab}^{i' j'}$ проводится ребро, если выполняются оба следующих условия:
    \begin{itemize}
        \item Вершины $w_{ab}^{i j}$ и $w_{ab}^{i' j'}$ соответствуют разным парам героев ($a \neq a' \lor b \neq b'$)
        \item Вершины $w_{ab}^{i j}$ и $w_{ab}^{i' j'}$ лежат в разных множествах из семейства $F_2$ ($i \neq i' \lor j \neq j'$)
    \end{itemize}
    
    Также проведём все рёбра между всеми вершинами множеств $F$ и $F_2$. Всем свежепроведённым рёбрам припишем вес 0. Очевидно, что любая максимальная клика содержит по одной вершине из каждого из множеств $V$ и $W$ семейств $\mathcal{F}$ и $\mathcal{F}_2$. 
    
    Припишем каждому ребру вида $(v_a^k, w_{ab}^{i j})$ или $(v_b^k, w_{ab}^{i j})$ некоторый высокий константный вес $N$. Данные рёбра содиняют вершину из семейства $\mathcal{F}$, соответствующую некоторому герою с вершиной из семейства $F_2$, соответствующей паре героев, куда этот герой входит. Покажем, что любая максимальная клика в таком графе, содержащая множество вершин из семейства $\mathcal{F}$, соответствующих некоторому набору героев, также содержит и набор вершин из семейства $\mathcal{F}_2$, соответствующий всем парам героев из этого набора.
    
    В данной клике, рёбра, соединяющие вершины из семейств $\mathcal{F}$ и $\mathcal{F}_2$ вносят суммарный вклад в вес, равный $2 \binom{n}{2} N$, т.к. в неё входит ровно $2 \binom{n}{2}$ рёбер с добавленным высоким константным весом $N$ --- по два, инцедентых каждой вершине из $\mathcal{F}_2$. Отметим, что при взятии в клику из семейства $\mathcal{F}_2$ любой вершины, не соответствующей паре взятых вершин из $\mathcal{F}$ среди рёбер клики будет менее двух рёбер с добавленным высоким константным весом $N$. Таким образом, данная клика не будет максимальной по весу.
    
Таким образом, утверждение доказано. Отметим, что добавление на рёбра весов, малых по сравнению с $N$ сохраняет истоинность утверждения. Поскольку $N$ берётся произвольно, можно считать, что все числовые значения сил и бонусов малы по сравнению с $N$.

Добавим тогда к весам рёбер вида  $(v_c^{k}, w_{a,b}^{i,j} )$ , соединяющих вершины из множеств 
$\mathcal{F}$ и $\mathcal{F}_2$ бонусы, которые альянс из пары героев под номерами $a$ и $b$ даёт герою под номером $c$. Тогда, если в выбранной команде есть герои $a$, $b$ и $c$, то в вес данной клики будет включён этот бонус. Поскольку во все рассматриваемые клики включено одинаковое количество рёбер с добавленным весом $N$, то максимальной будет та клика, где максимальной будет сумма сил героев (сумма весов рёбер между вершинами семейства $\mathcal{F}$) и бонусов (веса рёбер между вершинами семейств $\mathcal{F}$ и $\mathcal{F}_2$ без учёта констант $N$).
    
Таким образом, вес клики соответствует суммарному бонусу от команды, откуда и видно сведение.
    
\end{proof}


\subsection{Всё вместе. Небольшой итог}



Вариант задачи в версии "Да/Нет". 

Можно вставить про  то, как да/нет версия формулируется. И как с её помощью решается задача DU  за логарифмическую сложность методом бинарного поиска. Тем самым оставаясь в классе np-complete

\begin{theorem}
    Задача Dota Underloards задаваемая системой неравенств \ref{DUIP} принадлежит классу NP-полных задач.
\end{theorem}

\begin{proof}
	В предыдуших двух частях мы показали, что задача UnderLords сводится в одну и в другую сторону. Таким образом DU для фиксированного числа альянсов в версии "Да/Нет" принадлежит классу NP-полных задач.  таким образом задача DU NP-трудна, и при этом сам лежит в классе NP.
\end{proof}

\section{Практическое применение для реальной задачи Dota Underlords}
\label{SectionComputationalResults}
  
Мы применяем данную модель для анализа реальной задачи Dota Underlords. Отметим, что полученные результаты не стоит считать некоторой объективной оценкой качества команды героев. Причина состоит в неизбежном упрощении сил героев и влияния, которое оказывают альянсы. Каждый герой в Underlords обладает некоторой способностью, которая активируется при заполнении шкалы маны и обладает некоторым временем перезарядки. Способности и бонусы альянсов также весьма разнообразны по своему влиянию на игру - они могут наносить урон, лечить союзников, мешать врагам пользоваться своими способностями и прочее. К счастью, в игре есть система из пяти <<ярусов>>, устроенная так, что герои внутри яруса примерно равны по силе.

В рамках упрощённой модели мы принимаем следующее.

1) Силы всех героев первого яруса равны 1, второго 2, третьего - 3, чевертого - 4, пятого - 5
2) Альянсы дают один и тот же процентный бонус всем, на кого они одинаково влияют.
3) Бонус от альянса составляет примерно 10-30 процентов от силы героя.

Детализированная таблица может быть найдена в нашем \href{https://github.com/aponom84/UnderLords/blob/master/UnderLordsData.xlsx}{репозитории} \cite{UnderLoardsInput}.

Таблица альянсов \\

\begin{table}
\center
\resizebox{!}{10cm} {
\begin{tabular}{llrl}
\label{aliances}
{№} &                 Герой &  Сила &                       Альянсы \\
\midrule
0  &                 tusk &      1 &               savage, warrior  \\
1  &           venomancer &      1 &               scaled, summoner \\
2  &         shadow demon &      1 &               demon, heartless \\
3  &          drow ranger &      1 &    heartless, hunter, vigilant \\
4  &          bloodseeker &      1 &           blood-bound, deadeye \\
5  &         nyx assassin &      1 &               assassin, insect \\
6  &       crystal maiden &      1 &                    human, mage \\
7  &                 tiny &      1 &           primordial, warrior  \\
8  &             batrider &      1 &                  knight, troll \\
9  &               magnus &      1 &                  druid, savage \\
10 &             snapfire &      1 &                 brawny, dragon \\
11 &           arc warden &      1 &           primordial, summoner \\
12 &                razor &      1 &               mage, primordial \\
13 &               weaver &      1 &                 hunter, insect \\
14 &              warlock &      1 &  blood-bound, healer, warlock  \\
15 &               dazzle &      2 &                  healer, troll \\
16 &         earth spirit &      2 &               spirit, warrior  \\
17 &         storm spirit &      2 &                   mage, spirit \\
18 &         witch doctor &      2 &                troll, warlock  \\
19 &          bristleback &      2 &                 brawny, savage \\
20 &     legion commander &      2 &                champion, human \\
21 &        queen of pain &      2 &                assassin, demon \\
22 &     nature's prophet &      2 &                druid, summoner \\
23 &                 luna &      2 &               knight, vigilant \\
24 &           windranger &      2 &               hunter, vigilant \\
25 &            ogre magi &      2 &       blood-bound, brute, mage \\
26 &                pudge &      2 &            heartless, warrior  \\
27 &          beastmaster &      2 &                 brawny, hunter \\
28 &         chaos knight &      2 &                  demon, knight \\
29 &              slardar &      2 &               scaled, warrior  \\
30 &              abaddon &      3 &              heartless, knight \\
31 &                viper &      3 &               assassin, dragon \\
32 &           juggernaut &      3 &               brawny, warrior  \\
33 &         ember spirit &      3 &               assassin, spirit \\
34 &                   io &      3 &              druid, primordial \\
35 &         shadow fiend &      3 &                demon, warlock  \\
36 &                lycan &      3 &        human, savage, summoner \\
37 &          broodmother &      3 &               insect, warlock  \\
38 &            morphling &      3 &               mage, primordial \\
39 &          lifestealer &      3 &               brute, heartless \\
40 &           omniknight &      3 &          healer, human, knight \\
41 &          terrorblade &      3 &                  demon, hunter \\
42 &        shadow shaman &      3 &                summoner, troll \\
43 &               enigma &      3 &               primordial, void \\
44 &     treant protector &      3 &                   brute, druid \\
45 &                 doom &      4 &                   brute, demon \\
46 &            disraptor &      4 &               brawny, warlock  \\
47 &          void spirit &      4 &                   spirit, void \\
48 &               mirana &      4 &               hunter, vigilant \\
49 &           tidehunter &      4 &               scaled, warrior  \\
50 &            necrophos &      4 &    healer, heartless, warlock  \\
51 &           lone druid &      4 &        druid, savage, summoner \\
52 &                 sven &      4 &          human, knight, scaled \\
53 &                slark &      4 &               assassin, scaled \\
54 &     templar assassin &      4 &       assassin, vigilant, void \\
55 &  keeper of the light &      4 &                    human, mage \\
56 &                  axe &      5 &                  brawny, brute \\
57 &        faceless void &      5 &                 assassin, void \\
58 &            sand king &      5 &                 insect, savage \\
59 &                 lich &      5 &                heartless, mage \\
60 &               medusa &      5 &                 hunter, scaled \\
61 &        dragon knight &      5 &          dragon, human, knight \\
62 &        troll warlord &      5 &                troll, warrior  \\
\bottomrule
\end{tabular}
}
\caption{Таблица силы геров и принадлежности их к альянсам. }
\end{table}

%%%%%%%%%%%%%%%%%%%%%%%%%%%%%%%%%%%%%%%%%%%%%%
%  RESULT TABLE
%%%%%%%%%%%%%%%%%%%%%%%%%%%%%%%%%%%%%%%%%%%%%%

Таблица результатов выглядит следущим образом \ref{solution}:
\begin{table}
\resizebox{17cm}{!} {
\begin{tabular}{l| *{8}{p{1.6cm}} | *{3}{ p{1cm}} }
\label{solution}
{Герой} &                   &               &                &                &                &                  &                  &                  &  Вклад альянса &  Сила героев &   сумма \\
\midrule
broodmother   &  heartless 2 +0.3  &  human 2 +0.3  &  insect 2 +0.3  &  scaled 2 +0.6  &   troll 2 +0.3  &  warlock  2 +0.6  &  warlock  4 +0.6  &                   &                  3.0 &           3 &   6.0 \\
disraptor     &  heartless 2 +0.4  &  human 2 +0.4  &  insect 2 +0.4  &  scaled 2 +0.8  &   troll 2 +0.4  &  warlock  2 +0.8  &  warlock  4 +0.8  &                   &                  4.0 &           4 &   8.0 \\
dragon knight &  heartless 2 +0.5  &  human 2 +0.5  &  insect 2 +0.5  &  knight 2 +1.0  &  scaled 2 +1.0  &     troll 2 +0.5  &  warlock  2 +0.5  &  warlock  4 +0.5  &                  5.0 &           5 &  10.0 \\
lich          &  heartless 2 +0.5  &  human 2 +0.5  &  insect 2 +0.5  &  scaled 2 +1.0  &   troll 2 +0.5  &  warlock  2 +0.5  &  warlock  4 +0.5  &                   &                  4.0 &           5 &   9.0 \\
medusa        &  heartless 2 +0.5  &  human 2 +0.5  &  insect 2 +0.5  &  scaled 2 +1.0  &   troll 2 +0.5  &  warlock  2 +0.5  &  warlock  4 +0.5  &                   &                  4.0 &           5 &   9.0 \\
necrophos     &  heartless 2 +0.4  &  human 2 +0.4  &  insect 2 +0.4  &  scaled 2 +0.8  &   troll 2 +0.4  &  warlock  2 +0.8  &  warlock  4 +0.8  &                   &                  4.0 &           4 &   8.0 \\
sand king     &  heartless 2 +0.5  &  human 2 +0.5  &  insect 2 +0.5  &  scaled 2 +1.0  &   troll 2 +0.5  &  warlock  2 +0.5  &  warlock  4 +0.5  &                   &                  4.0 &           5 &   9.0 \\
sven          &  heartless 2 +0.4  &  human 2 +0.4  &  insect 2 +0.4  &  knight 2 +0.8  &  scaled 2 +0.8  &     troll 2 +0.4  &  warlock  2 +0.4  &  warlock  4 +0.4  &                  4.0 &           4 &   8.0 \\
troll warlord &  heartless 2 +0.5  &  human 2 +0.5  &  insect 2 +0.5  &  scaled 2 +1.0  &   troll 2 +1.0  &  warlock  2 +0.5  &  warlock  4 +0.5  &                   &                  4.5 &           5 &   9.5 \\
witch doctor  &  heartless 2 +0.2  &  human 2 +0.2  &  insect 2 +0.2  &  scaled 2 +0.4  &   troll 2 +0.4  &  warlock  2 +0.4  &  warlock  4 +0.4  &                   &                  2.2 &           2 &   4.2 \\
\bottomrule
\end{tabular}
}
\caption{Оптимальный состав команды в для игры Dota UnderLoards. Таблица результатов }
\end{table}


\section{Заключение}
\label{SectionConclusion}
По традиции мы завершаем нашу статью разделом "заключение". 
В работе мы продемонстрировали к 
Результаты и Jupyter-тетради могут быть найдены в репозитории по адресу. 
Мы надеемся, что наша статья о популярной видео-игре  поможет привлеч внимание молодых умов к целочисленному программировению, методам дискретной оптимизации, а также к проблеме тысячилетия P=?NP. 

И важно, что мы показали что поиск такого-то оптимального (записать формальным языком)

Важно, что математическая постановка задачи задаваемая набором неравенств (2) может  рассматриваться сама по себе, абстрагируясь от предметной области. И в данной работе показано, что кажущаяся на первый взгляд np-hard задача, всё-таки является np-полной. 
что это также вклад в исследование np-послных задач. 
что сама по себе задача описанная неравенствами 

\bibliographystyle{unsrt}
\bibliography{references}


\end{document}




%
%\section{Частные разрешимые случаи}
%
%Как и многие подобные задачи, задача Dota Underlords становится полиномиально разрешимой при определённых ограничениях. 
%
%Рассмотрим простейший нетривиальный случай
%
%\begin{enumerate}
%    \item Каждый герой принадлежит ровно двум альянсам
%    \item Каждый альянс содережит ровно двух героев
%    \item Сила каждого героя равна $a$
%    \item Бонус от каждого альянса на каждого героя одинаков и равен $\frac{b}{2}$
%\end{enumerate}
%
%Таким образом, суммарный бонус от сбора альянса в данной модели равен $b$. 
%
%\begin{theorem}
%    Задача Dota Underlords в указанной формулировке решается за $O(nb)$, если $b$ - натуральное и NP-полна, если числа $a$ и $b$ - несравнимы. 
%\end{theorem}
%
%\begin{proof}
% Данную задачу удобно перефомулировать на языке теории графов. Обозначим героев за вершины графа, а отношение "находятся в одном альянсе" - за ребро. Тогда в данном графе степени всех вершин равны 2 и он разбивается на несколько (возможно один) циклов, несвязных между собой. Т.к. число элементов в искомом наборе фиксировано, то итоговая целевая сумма зависит от числа рёбер, порождаемых в данном графе множеством взятых вершин-героев. Таким образом, если мы берём из некоторого цикла множество вершин мощности $k$, которое не совпадает с множеством всех вершин цикла, то мы можем получить самое большее бонус $(k-1)b$ - взяв вершины, порождающие путь в данном цикле. В случае, если мы берём весь цикл, то мы получаем бонус $kb$. таким образом, для достижения наибольшего бонуса необходимо, чтобы все взятые вершины порождали в графе некоторое множество циклов (возяожно ноль) и не более одного пути. При этом, случай, когда пути вообще нет лучше случая, когда он есть. В случае, если нам надо взять всего $n$ элементов и мы смогли взять $n$ вершин, порождающих исключительно циклы, мы получаем итоговую сумму $n(a+b)$. Если же присутствует путь, то мы получаем сумму $na+(n-1)b$. Отметим, что фактически необходимо проверить возможность построения набора, порождающих исключительно циклы, т.к. построить набор, порождающий циклы и путь --- элементарно, достаточно добавлять циклы в набор в произвольнос порядке, пока не окажется, что мы хотим добавить цикл размера большего, чем оставшееся количество вершин. Тогда вместо этого добавляем произвольный путь из этого цикла. 
% 
% Данная задача легко сводится к задаче о рюкзаке с целыми весами и решается методом динамического программирования. Действительно, размеры циклов в этой задаче соответсвуют весам и объёму элементов, общее их количество в итоговом решении - объёму рюкзака. Таким образом, наобходимо лишь проверить, что в решении данного экземпляра задачи о рюкзаке рюкзак заполнится полностью.
% 
% Сложность алгоритма  в данной задаче равна $O(nb)$ при целом $b$. Задача тогда решается методом динамического программирования.
% 
% В случае произвольного $b$ данная задача представлет собой специфический случай задачи о сумме. Задачей о сумме называется задача в которой требуется определить, есть ли в данном множестве чисел подмножество с некоторой суммой $s$ (в каноническом случае - с суммой 0). Известно, что он - NP-полна Таким образом, задача Dota Underlords сводится к ней и значит NP-сложна. 
%\end{proof}



%процитировать классиков Cook, S.A. (1971). "The complexity of theorem proving procedures". Proceedings, Third Annual ACM Symposium on the Theory of Computing, ACM, New York. pp. 151–158. doi:10.1145/800157.805047.
% Cobham, Alan (1965). "The intrinsic computational difficulty of functions". Proc. Logic, Methodology, and Philosophy of Science II. North Holland.

%процитировать  другие видео-игры
%https://www.ics.uci.edu/~eppstein/cgt/hard.html
%https://arxiv.org/abs/cs/0210020
%https://web.archive.org/web/20061216121200/http://for.mat.bham.ac.uk/R.W.Kaye/minesw/ordmsw.htm



